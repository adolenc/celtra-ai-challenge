\section{Zaključek} 

V poročilu smo opisali pristop, ki se je na javni lestvici izkazal za
najboljšega. Žal pa ne pride brez pomanjkljivosti: kjer pravilno napovemo
uporabnikov zorni kot bo sicer uporabnik prejel idealno kvaliteto, kolikor
hitro pa se odloči za preveč odtavati od naše napovedi, pa bo prejel kvaliteto
$1\%$.

Naš pristop najverjetneje tudi ni najboljši kar se tiče racionalizacije. Kljub
temu, da smo z mahanjem rok lahko na nek način upravičili njegovo zasnovo,
v resnici deluje precej drugače kot pa izgleda dejanski ogled videa s strani
uporabnika. S tem ko računamo poti, ki maksimizirajo energijo, namreč naš
algoritem `vidi v prihodnost' in lahko pove, kje se bodo zgodile najbolj
zanimive scene. Po drugi strani uporabnik tega ne more nujno storiti in obrača
glavo le na podlagi trenutne slike.
