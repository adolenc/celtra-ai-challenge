\section{Uvod} 

Videi so eni najpomembnejših in najpopularnejših medijev v uporabi, tako za
prenašanje pomembnih informacij kot tudi za zabavo.  Z ogromnimi napredki na
področju virtualne realnosti v zadnjih nekaj letih lahko pričakujemo, da bodo
tudi tu videi igrali zelo pomembno vlogo, vendar pa novo področje s sabo
prinese nove izzive. V virtualni realnosti namreč pogosto omogočimo uporabniku,
da poljubno premika svojo glavo in ker mu ne želimo uničiti občutka imerzije,
mu predvajamo video za vseh $360\degree$. Da pa mu to omogočimo, mu
moramo pošiljati ogromno količino podatkov naenkrat če mu želimo zagotoviti
dobro kvaliteto posnetka za vseh $360$ stopinj. Zato je morda interesantno,
da vnaprej predvidimo v katero smer bo uporabnik gledal ter mu pošljemo visoko
kvaliteto le za predviden zorni kot.

V nalogi tako poskusimo na podlagi prejšnjih ogledov drugih uporabnikov nekega
videa ter na podlagi nekaj sekundne zgodovine obračanja glave uporabnika
napovedati, katere tokove se nam splača poslati, da mu zagotovimo čim boljšo
izkušnjo.

V poročilu najprej predstavimo podane podatke ter omejitve naloge, nato opišemo
in utemeljimo našo rešitev, za konec pa predstavimo še nekaj pomankljivosti
naše naloge ki bi prišle v poštev v dejanski uporabi.
